
\section{基本元素}

在Blender中,一个基本的Ballance关卡至少需要包括:

\begin{enumerate}

\item 一个开头的四盘点火焰(注意是火焰,不是底部路面,路面不是必须的)。

\item 一个重生点(第一小节的)。

\item 一个飞船。

\end{enumerate}

如果不包含以上物体,即使导出了关卡,Ballance也会拒接加载。

此外,对于只有一小节的地图,Ballance在渲染开头的四盘点火焰的时候会出现Bug,表现为火焰粒子呈现为不自然的三角形。为了避免这种情况,请添加第一小节的检查点(两火焰盘点)和第二小节的重生点,让关卡小节数变为2。这就是所谓的一小节关卡Bug。我们不建议制作只有一小节的关卡,尽管它玩起来没有任何问题。

\section{钢轨参数}

Ballance中的钢轨看起来是圆形,但实际上其截面是八边形,建模时切勿随意选择边数。

单轨截面的半径为0.35。双轨两根钢轨之间的间距是3.75(这似乎不是精确值,是大家都在用的所谓约定值)。单双轨结合处,单轨相对于双轨中心在Z轴上下降0.61707(由失衡之梦计算,取玩家球半径为2)。

普通侧轨倾斜角度79.563$^\circ$,与此同时双轨之间的间距要改为3.864(由BallanceBug提供)。螺旋轨两层之间的间距为5(从Level 10测量)。侧边螺旋轨两层之的间距(实际上是每一次迭代的上升高度,因为侧边螺旋轨的轨道为两层共用的)为3.6(由Level 13结尾,Level 9开头测量)。

\section{禁止缩放物理化的物体}

由于Ballance的物理引擎的限制,严禁在物理化的物体上(归入任意物理化组,即Phys\_Floors,Phys\_FloorRails,Phys\_FloorStopper任意之一的物体)使用缩放(移动和旋转没有问题),否则游戏中渲染的效果和实际碰撞体积会不重合(渲染是正确的,是碰撞体积错了)。

为了消除缩放参数,可以在Blender中使用Alt + S消除物体的缩放,如果实在需要缩放物体,应该在编辑模式下缩放,或物体模式下缩放后使用Object - Apply - Scale将缩放应用到网格上,而不是保留在3D物体上。

\section{死亡区}

死亡区是Ballance游戏中的一种物体,当玩家球碰到它的时候就会死亡,回到重生点。

死亡区的判定是按照其Bounding Box(碰撞盒)来判断的,只要进入其碰撞盒,就判断和死亡区接触。。死亡区并不是指物体触碰到死亡区的面才产生效果的,不要有这样的误解。所以不要把死亡区本身塑造成歪七扭八的,并指望它能工作。相反,应该用多个死亡区来达成相同效果。

死亡区是可以有缩放的,这与那些需要物理化的物体不同。但旋转死亡区可能带来意料之外的效果(旋转后的死亡区的碰撞盒不再是死亡区形状本身了,不方便观察),不要那么做。

死亡区需要隐藏,不然会在游戏里显示,很难看。在Blender里隐藏就是在Virtools,以及地图中隐藏了,二者隐藏属性是相通的。

\section{贴图文件的保存格式}

\subsection{Save Options}

贴图的Save Options控制了贴图如何被保存在Ballance地图中。为了尽可能减少地图文件大小,且保证地图不会出现找不到贴图从而渲染错误的情况,请遵循以下要求设置贴图的Save Options:

\begin{itemize}

\item 对于所有Ballance原版贴图(例如路面贴图,钢轨贴图,木板贴图等,凡是能在Texture文件夹中找到的都是),使用External保存格式。

\item 对于所有其它的贴图(例如你自己添加的贴图,就比如要素超载里面的那个滑稽柱子顶面就是),使用Raw Data保存格式。

\item \textbf{非常不建议}使用Use Global,因为根据用户全局选项的不同,这个保存方式会造成歧义。除非是在修复一些老旧地图,或者你知道你在做什么的情况下,否则不要使用它。

\end{itemize}

\subsection{Video Format}

Video Format决定了你的贴图在游戏运行时是怎么传递给GPU的(大概那么理解就好)。为了防止贴图出错,尤其是透明度出错,且最小化显示压力,请遵循如下要求来设置贴图的Video Format:

\begin{itemize}

\item 对于非透明贴图(例如路面贴图等),请使用16 Bits ARGB1555。

\item 对于有透明度的贴图(例如柱子下班部分的透明,风扇的网罩),请使用32Bits ARGB8888。

\item \textbf{非常不建议}使用其它任何Video Format(因为我也没用过)。

\end{itemize}

\section{光源}

BBP插件不支持光源的导入导出,但可能在未来支持。

也因为于此,一些使用自定义光源的Ballance地图导入Blender后会丢失地图中的光源,同时再保存这些光源也不会回来。

\section{Stopper没有声音}

自制地图中的Stopper即使正确归组后,仍然没有声音的问题,主要是由于Ballance的一个Bug导致的。由于Ballance原版脚本编写的失误,只有位于Phys\_FloorStopper组中的第一个物体才会发出声音。

所以为了让地图中所有Stopper都有声音,解决方案是在地图发布前将所有Stopper合并成一个物体(建议在地图发布前做是因为合并了后的Stopper,如果想再分别移动,操作上会显得比较麻烦,而且还可能会误操作),即在Blender中选中所有Stopper然后Ctrl + J即可。

\section{路面影子问题}

路面影子问题曾是困扰Ballance制图界许久的一个问题,该问题在正式发现原理前的解决方案是从原版关卡中获取3D物体,然后将我们的物体的网格赋予这个3D物体。这样的反复操作显得非常的蠢。在原理发现后,可由Zzq的插件或yyc12345的插件在Virtools中一键修复。但这还是需要处理的。

使用BBP导出的Ballance地图完全不需要担心影子问题,只要将需要显示影子的物体归入Shadow组中,即可在游戏中获得正常的影子显示。对于其它缺少影子的地图(指的是那些已经归了Shadow组,但是没有影子产生的地图),也可以通过将其导入Blender中后再直接导出,简单地进行修复。

与柱子不断问题不同的是,即使经过Virtools的重新导入导出,路面的影子也不会消失,所以可以放心地使用Virtools再编辑。

\section{柱子不断问题}

柱子不断问题曾是困扰Ballance制图界许久的一个问题,该问题在本插件面世之前的最终解决方案是通过chirs241097所发明的地图内嵌脚本来解决的。但这对于制图新手相当麻烦。

通过BBP导出的Ballance地图,只要操作合理(你没去乱调材质和贴图设置的话),其柱子是保证不断的,不需要特殊处理。通过BBP导出的地图的行为更贴近为原版关卡的行为,即只要通过Virtools再打开并保存一次,柱子就会断掉。所以如果你的地图不涉及使用Virtools再编辑(比如添加脚本之类的),那导出的地图就可以直接发布游玩。如果你使用了Virtools进行编辑,那就必须要遵循chirs241097所编写的教程,编写地图内嵌脚本,并修改柱子贴图文件的Video Format。

与柱子不断问题相似的还有灯笼的黄色光芒,风扇网格的透明等问题。这些问题和柱子不断一样,只要不涉及Virtools重新保存,就不需要特别关注它们。

\section{无用物体清理}

Blender不是Virtools,所以BBP导出的时候只会导出你选择的那些3D物体,不会产生无用物体,这与Virtools不同。在Virtools中,无用物体需要在一个特殊的窗口中清理,如果不清理,这些无用物体也会跟着一起保存到地图文件中。

Blender下,没在地图中用到的物体天然地不会被BBP保存到地图中。如果你是想清理Blender本身中的无用物体,请使用File - Clean Up - Purge Unused Data来达成。
