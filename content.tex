
\section{Stopper没有声音}

自制地图中的Stopper即使正确归组后,仍然没有声音的问题,主要是由于Ballance的一个Bug导致的。由于Ballance原版脚本编写的失误,只有位于Phys\_Stopper组中的第一个物体才会发出声音。

所以为了让地图中所有Stopper都有声音,解决方案是在地图发布前将所有Stopper合并成一个物体(建议在地图发布前做是因为合并了后的Stopper,如果想再分别移动,操作上会显得比较麻烦,而且还可能会误操作),即在Blender中选中所有Stopper然后Ctrl + J即可。

\section{路面影子问题}

路面影子问题曾是困扰Ballance制图界许久的一个问题,该问题在正式发现原理前的解决方案是从原版关卡中获取3D物体,然后将我们的物体的网格赋予这个3D物体。这样的反复操作显得非常的蠢。在原理发现后,可由Zzq的插件或yyc12345的插件在Virtools中一键修复。但这还是需要处理的。

使用BBP导出的Ballance地图完全不需要担心影子问题,只要将需要显示影子的物体归入Shadow组中,即可在游戏中获得正常的影子显示。对于其它缺少影子的地图(指的是那些已经归了Shadow组,但是没有影子产生的地图),也可以通过将其导入Blender中后再直接导出,简单地进行修复。

与柱子不断问题不同的是,即使经过Virtools的重新导入导出,路面的影子也不会消失,所以可以放心地使用Virtools再编辑。

\section{柱子不断问题}

柱子不断问题曾是困扰Ballance制图界许久的一个问题,该问题在本插件面世之前的最终解决方案是通过chirs241097所发明的地图内嵌脚本来解决的。但这对于制图新手相当麻烦。

通过BBP导出的Ballance地图,只要操作合理(你没去乱调材质和贴图设置的话),其柱子是保证不断的,不需要特殊处理。通过BBP导出的地图的行为更贴近为原版关卡的行为,即只要通过Virtools再打开并保存一次,柱子就会断掉。所以如果你的地图不涉及使用Virtools再编辑(比如添加脚本之类的),那导出的地图就可以直接发布游玩。如果你使用了Virtools进行编辑,那就必须要遵循chirs241097所编写的教程,编写地图内嵌脚本,并修改柱子贴图文件的Video Format。

与柱子不断问题相似的还有灯笼的黄色光芒,风扇网格的透明等问题。这些问题和柱子不断一样,只要不涉及Virtools重新保存,就不需要特别关注它们。
