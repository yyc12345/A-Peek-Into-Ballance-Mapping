\section*{序言}
%Star mean this section is not included in content

本文档是针对使用Blender自制Ballance地图的补充说明。是给那些已经了解Ballance基本内容,Blender建模技巧的人,一个制图方面的小补充,让他们可以正常地产出一张Ballance自制地图。因为做制图视频教程太麻烦了,得照顾完全初学者的制图人,很多概念都得从头讲,导致视频时长过长又没有干货。所以我抽空写了这样一个短小精悍的文档给那些有相关基础的人,让他们可以快速入门。

我在书写本文档的时候,会假定您已经对一些概念有所了解,不会在文档中对这些概念进行解释。所以阅读此文档前,您需要:

\begin{enumerate}

\item 了解Ballance,玩过Ballance,知道Ballance里有什么内容(例如当我提到T板的时候,您应当立刻在脑海中想象出它的样子,而不是什么都不知道)。

\item 已经会使用Blender进行基础的建模,材质设置等。对3D相关名词(物体,网格,它们各自承担什么责任等)有初步的了解。

\item 已经完整地阅读Ballance制图用插件,BallanceBlenderPlugin的帮助文档(会熟练地安装,卸载,更新插件;知道插件的各个功能都在哪里;知道如何向插件作者汇报错误)。

\end{enumerate}

如果您不满足上述任意一个条件,那么我不建议您阅读此文档。除非您确信您是一个有充足探究精神的人,愿意去阅读枯燥的文档,愿意去了解未知的概念,甚至愿意踏入没有任何帮助支持的领域。因为这样的您会在遇到搞不懂的概念时立刻尝试去搞懂它,而不是怨天尤人地在那抱怨本文写的不够详细。

总而言之,这个文档是写给那些\textbf{聪明人}看的,是写给那些只需要一些点拨,就可以触类旁通,由点到面的人的。而不是给那些慵懒的,无所谓的,总想着抄捷径的人准备的。
